\documentclass[14pt]{extarticle}
\usepackage{fontspec}
\usepackage[ukrainian]{babel}
\usepackage[contents=TODO, color=gray, pages=some]{background}
\usepackage[a4paper, left=20mm, top=20mm, right=10mm, bottom=20mm]{geometry}
\usepackage{lipsum}
\usepackage{indentfirst}
\usepackage{titlesec}
\usepackage{stringstrings}
\usepackage{fancyhdr}
\usepackage{enumitem}
\usepackage{tabularx}
\usepackage[hidelinks]{hyperref}
\usepackage{textcase}

% TODO: make section uppercase in embedded links

% Uppercase sections in ToC
\makeatletter
\let\oldcontentsline\contentsline
\def\contentsline#1#2{%
  \expandafter\ifx\csname l@#1\endcsname\l@section
    \expandafter\@firstoftwo
  \else
    \expandafter\@secondoftwo
  \fi
  {%
    \oldcontentsline{#1}{\MakeTextUppercase{#2}}%
  }{%
    \oldcontentsline{#1}{#2}%
  }%
}
\makeatother

\hypersetup{linktoc=all}

\pagestyle{fancy}
\fancyhf{}
\fancyhead[R]{\thepage}
\renewcommand{\headrule}{}
\setlength{\headheight}{17pt}

\setmainfont{Times New Roman}
\linespread{1.5}
\setlength{\parindent}{16mm}

\titleformat{\section}
  { \filcenter % center
    \fontsize{14pt}{14pt}
    \bfseries % semi bold
    \MakeUppercase
  }
  {\thesection~}
  {0pt}
  {}
\titlespacing*{\section}{0pt}{12pt}{6pt}

\titleformat{\subsection}
  { \fontsize{14pt}{14pt}
  }
  % TODO: sentence case
  {\thesubsection~}
  {0pt}
  {}
\titlespacing*{\subsection}{0pt}{6pt}{0pt}

% all sections will start at new page
\let\oldsection\section
\renewcommand{\section}{\clearpage\oldsection}

\newcommand{\unnumberedSection}[1]{%
  \section*{#1}%
  \phantomsection
  \addcontentsline{toc}{section}{#1}%
}

\newcommand{\unnumberedSubsection}[1]{%
  \subsection*{#1}%
  \addcontentsline{toc}{subsection}{#1}%
}

\begin{document}


титульний аркуш
\BgThispage
\thispagestyle{empty}

\newpage
завдання
\BgThispage
\section*{реферат}
\BgThispage

% зміст
\tableofcontents

\unnumberedSection{ПЕРЕЛІК УМОВНИХ ПОЗНАК, СИМВОЛІВ, СКОРОЧЕНЬ І ТЕРМІНІВ}
УДК --- Універсальна десяткова класифікація

НР --- наукова робота

% ommited because it's only used for groupping
% основну частину:
  \unnumberedSection{ВСТУП}  
  Задача, яку розглядає дана робота ---
  переклад та перевірка точності перекладу текстів (НР)
  з натуральної мови на формальну (шифр УДК).
  
  Точність УДК шифрів є ключовою для ефективної класифікації
  та пошуку наукової літератури.
  Однак, ручний вибір та перевірка УДК шифрів вимагає значних зусиль,
  займає багато часу та супроводжується високим ризиком виникнення помилок.
  Зовнішня перевірка може зменшити ризик помилок, але вимагає додаткових зусиль.
  Натомість, автоматизоване рішення може значно зменшити час та ресурси,
  необхідні для вибору та перевірки, а також мінімізувати ризик помилок.
  Тому ця робота має на меті розробити надійний
  та ефективний автоматизований інструмент для вибору
  та перевірки УДК шифрів НР.

  % TODO: move this part to some other chapter
  \section*{TODO: move to other section}
  \BgThispage
  Перевірка може бути виконана у два кроки:
  \begin{enumerate}[labelindent=\dimexpr\parindent*2\relax, leftmargin=*]
    \item Переклад
    \item Порівняння наданого шифру із перекладом.
  \end{enumerate}

  Переклад у свою чергу вимагає попереднього стискання
  вихідного тексту для виключення надмірностей.
  Через розміри та семантику шифрів,
  стисненний результат має представляти собою дуже короткий (декілька речень)
  текст, який перераховує теми оригінального тексту та їх співвідношення.

  % ommited because it's only used for groupping
  % основний текст кваліфікаційного проєкту (роботи):

  % TODO: add bibliographic references
  \section{ЗБІР ТА АНАЛІЗ ВИМОГ}

  \subsection{Універсальна десяткова класифікація}
  Універсальна десяткова класифікація (УДК) —--
  бібліографічна та бібліотечна класифікація,
  представляє систематичне впорядкування всіх галузей людських знань,
  організованих як узгоджена система,
  у якій галузі знань заємопов’язані.
  
  В УДК використовуються арабські цифри в десятковому порядку.
  Кожне число розглядається як десятковий дріб
  з опущеною початковою десятковою крапкою, яка визначає порядок запису.
  Для зручності читання УДК зазвичай ставиться розділовий знак
  після кожної третьої цифри.

  Основні таблиці містять різні дисципліни та галузі знань,
  розташовані в 9 основних класах, пронумерованих від 0 до 9
  (при цьому 4 клас є вільним).
  На початку кожного класу також є серія спеціальних допоміжних слів,
  які виражають аспекти, що повторюються в цьому конкретному класі.
  Основні таблиці в УДК містять понад 60 000 підрозділів.
  \begin{enumerate}
      [labelindent=\dimexpr\parindent*2\relax, leftmargin=*, start=0]
    \item Наука і знання.
    Організація.
    Комп'ютерна наука.
    Інформатика.
    Документація.
    Бібліотечна справа.
    Заклади.
    Публікації
    \item Філософія. Психологія
    \item Релігія. Теологія
    \item Суспільствознавство
    \item -
    \item Математика. Природничі науки
    \item Прикладні науки. Медицина, Техніка
    \item Мистецтво. Розваги. Спорт
    \item Мовознавство. Література
    \item Географія. Історія
  \end{enumerate}

  Загальні допоміжні засоби — концепції,
  які можна використовувати в поєднанні
  з будь-яким іншим кодом УДК з основних класів
  або з іншими загальними допоміжними засобами.
  Вони мають унікальні позначення, що виділяють їх у складних виразах.
  Звичайні допоміжні числа завжди починаються з певного символу,
  напр. = (знак рівності) завжди вводить поняття,
  що представляють мову документа;
  (0...) цифри в круглих дужках, починаючи з нуля,
  завжди представляють поняття, що позначає форму документа.
  Таким чином (075) підручник і =111 англійська мова може бути об’єднана,
  щоб виразити, наприклад, (075)=111 підручники англійською мовою,
  і в поєднанні з числами з основних таблиць
  УДК їх можна використовувати таким чином:
  2(075)=111 підручники з релігії англійською мовою,
  51(075)=111 Підручники з математики англійською мовою та ін.

  \begin{tabularx}{\dimexpr\linewidth - \parindent\relax}{cX}
    =... & Загальні допоміжні засоби мови. \\
    (0...) & Загальні допоміжні форми форми. \\
    (1/9) & Загальні допоміжні слова місця. \\
    (=...) & Загальні допоміжні ознаки людського походження,
    етнічної групи та національності. \\
    "..." & Загальні допоміжні слова часу,
    допомагає зробити хвилинний поділ часу, наприклад: "1993-1996" \\
    -0... & Загальні допоміжні характеристики загальних характеристик:
    Властивості, Матеріали, Відносини/Процеси та Особи. \\
    -02 & Загальні допоміжні властивості. \\
    -03 & Загальні допоміжні матеріали. \\
    -04 & Загальні допоміжні елементи відносин, процесів і операцій. \\
    -05 & Загальні допоміжні особи та особисті характеристики. \\
  \end{tabularx}

  Доступно кілька сполучних символів для зв’язку та розширення номерів УДК:

  \begin{tabular}{|l|l|}
    \hline
    Символ & Значення \\
    + & узгодження, доповнення \\
    / & послідовне розширення \\
    : & відношення \\
    $[~~]$ & підгрупування \\
    $*$ & Впроваджує нотацію, відмінну від УДК \\
    A/Z & Пряма алфавітна специфікація \\
    \hline
  \end{tabular}

  \subsection{Аналіз проблеми}
  Мета роботи ---
  розробити ПЗ для підбору та перевірки коректності УДК шифрів НР.
  Під підбором розуміється наступний функціонал --- на вході маємо текст роботи,
  на виході --- список класів та підкласів УДК (далі --- класів),
  до яких ця робота належить.
  Перевірка --- на вході маємо текст роботи та шифр,
  на виході --- згенерований шифр та оцінку схожості наданого
  та згенерованого шифрів, яким саме чином ця оцінка буде отримана,
  та як вона буде виглядати розглянемо в наступних розділах.
  
  Програму можна узагальнити до такої специфікації ---
  на вході маємо текст роботи та опціонально шифр,
  на виході маємо список класів та, якщо був наданий шифр,
  порівняльну оцінку зі згенерованими класами.

  Через те що наукові роботи можуть бути доступні у різноманітних форматах,
  варто уточнити що ПЗ буде приймати їх
  у форматі plain-text на англійській мові.

  Це можна розглядати як проблему класифікації --- маємо множину об'єктів,
  які певним чином розподілені на класи (підмножини).
  Задача класифікації ---
  різновид машинного навчання (англ. machine learning, далі —-- ML).
  Розрізняють два типи навчання:
  \begin{itemize}[labelindent=\dimexpr\parindent*2\relax, leftmargin=*]
    \item Навчання з учителем (англ. supervised learning) ---
    комп'ютеру надається набір даних (data set),
    в якому даним вже надано бажаний результат ---
    саме за ним комп'ютер і буде навчатись.

    \item Навчання без учителя (англ. unsupervised learning) ---
    набір даних, наданий комп'ютеру, ніяк не помічається ---
    комп'ютер сам має розбити дані на деякі групи (clustering).
  \end{itemize}

  Класи заздалегідь відомі, тож навчання без учителя, в нашому випадку,
  не підходить. Крім того можна використати існуючи класифіковані роботи,
  як набір даних для навчання.
  Але отримання достатнього набору даних вимагає великих зусиль ---
  потрібно отримати декілька робіт для кожного класу та підкласу
  та їх різних комбінацій. Тож пропонується простіший алгоритм:
  \begin{enumerate}[labelindent=\dimexpr\parindent*2\relax, leftmargin=*]
    \item Витягнемо з тексту роботи ключові слові
    \item Порівнюємо ключові слова із термінами та поняттями у каталозі УДК
  \end{enumerate}

  Останній шаг є дещо проблематичним
  через те що доступ до офіційного каталогу здійснюється на платній основі ---
  такий варіант не підходить для даного випадку через відсутність фінансування.
  Але існують безкоштовні та/або відкриті каталоги,
  які можна використати замість платної версії.
  Авжеж вони мають свої недоліки, наприклад: менша частота оновлень,
  менший корпус інформації, тощо.
  Не дивлячись на ці недоліки
  вони є достатнім ресурсом для початкової версії ПЗ.
  Також за бажанням ПЗ може бути відредагованим
  у майбутньому щоб використовувати платну версію.

  \newpage
  \subsection{Висновки}
  \lipsum[1-3]
  \BgThispage

  \section{ПРОЄКТУВАННЯ}
  \subsection{Зовнішнє проєктування}
  \subsection{Внутрішнє проєктування}
  \subsection{Проєктування архітектури системи}
  \subsection{Висновки}
  \BgThispage

  \section{РОЗРОБКА ПРОГРАМИ}
  \subsection{Висновки}
  \BgThispage

  \section{ТЕСТУВАННЯ ТА НАЛАГОДЖЕННЯ}
  \subsection{Висновки}
  \BgThispage

  \unnumberedSection{ЗАГАЛЬНІ ВИСНОВКИ ТА РЕКОМЕНДАЦІЇ}
  \BgThispage

  \unnumberedSection{СПИСОК ВИКОРИСТАНОЇ ЛІТЕРАТУРИ}
  \BgThispage

  \unnumberedSection{ДОДАТКИ}
  \BgThispage

\end{document}
