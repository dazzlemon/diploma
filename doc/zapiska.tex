\documentclass{article}
\usepackage{tikz}
\usepackage{eso-pic}
\usepackage{fontspec}
\usepackage[ukrainian]{babel}

\setmainfont{Times New Roman}

% all sections will start at new page
\let\oldsection\section
\renewcommand{\section}{\clearpage\oldsection}

\newcommand{\unnumberedSection}[1]{%
  \section*{#1}%
  \addcontentsline{toc}{section}{#1}%
}

\begin{document}

титульний аркуш
завдання
\section*{реферат}

% зміст
\tableofcontents

\unnumberedSection{перелік умовних познак, символів, скорочень і термінів}

% ommited because it's only used for groupping
% основну частину:
  \unnumberedSection{вступ}

  % ommited because it's only used for groupping
  % основний текст кваліфікаційного проєкту (роботи):

  \section{збір та аналіз вимог}
  \subsection{висновки}

  \section{проєктування}
  \subsection{висновки}

  \section{зовнішнє проєктування}
  \subsection{висновки}

  \section{внутрішнє проєктування}
  \subsection{висновки}

  \section{проєктування архітектури системи}
  \subsection{висновки}

  \section{розробка програми}
  \subsection{висновки}

  \section{тестування та налагодження}
  \subsection{висновки}

  \unnumberedSection{загальні висновки та рекомендації}

  \unnumberedSection{список використаної літератури}

  \unnumberedSection{додатки}

\end{document}
