\documentclass{article}
\usepackage{fontspec}
\usepackage[ukrainian]{babel}
\usepackage[contents=TODO, color=gray, pages=some]{background}
\usepackage[a4paper, left=20mm, top=20mm, right=10mm, bottom=20mm]{geometry}
\usepackage{lipsum}
\usepackage{indentfirst}
\usepackage{titlesec}

\setmainfont{Times New Roman}
\linespread{1.5}
\setlength{\parindent}{16mm}

\titleformat{\section}
  { \filcenter % center
    \fontsize{14pt}{14pt}
    \bfseries % semi bold
    \MakeUppercase
  }
  {\thesection~}
  {0pt}
  {}
\titlespacing*{\section}{0pt}{12pt}{6pt}

% all sections will start at new page
\let\oldsection\section
\renewcommand{\section}{\clearpage\oldsection}

\newcommand{\unnumberedSection}[1]{%
  \section*{#1}%
  \addcontentsline{toc}{section}{#1}%
}

\begin{document}


титульний аркуш
\BgThispage

\newpage
завдання
\BgThispage
\section*{реферат}
\BgThispage

% зміст
\tableofcontents

\unnumberedSection{перелік умовних познак, символів, скорочень і термінів}
\BgThispage

% ommited because it's only used for groupping
% основну частину:
  \unnumberedSection{вступ}  
  \lipsum[1-3]
  \BgThispage

  % ommited because it's only used for groupping
  % основний текст кваліфікаційного проєкту (роботи):

  \section{збір та аналіз вимог}
  \subsection{висновки}
  \BgThispage

  \section{проєктування}
  \subsection{висновки}
  \BgThispage

  \section{зовнішнє проєктування}
  \subsection{висновки}
  \BgThispage

  \section{внутрішнє проєктування}
  \subsection{висновки}
  \BgThispage

  \section{проєктування архітектури системи}
  \subsection{висновки}
  \BgThispage

  \section{розробка програми}
  \subsection{висновки}
  \BgThispage

  \section{тестування та налагодження}
  \subsection{висновки}
  \BgThispage

  \unnumberedSection{загальні висновки та рекомендації}
  \BgThispage

  \unnumberedSection{список використаної літератури}
  \BgThispage

  \unnumberedSection{додатки}
  \BgThispage

\end{document}
