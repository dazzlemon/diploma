\documentclass[14pt]{extarticle}
\usepackage{fontspec}
\usepackage[ukrainian]{babel}
\usepackage[contents=TODO, color=gray, pages=some]{background}
\usepackage[a4paper, left=20mm, top=20mm, right=10mm, bottom=20mm]{geometry}
\usepackage{lipsum}
\usepackage{indentfirst}
\usepackage{titlesec}
\usepackage{stringstrings}
\usepackage{fancyhdr}
\usepackage{enumitem}

\pagestyle{fancy}
\fancyhf{}
\fancyhead[R]{\thepage}
\renewcommand{\headrule}{}
\setlength{\headheight}{17pt}

\setmainfont{Times New Roman}
\linespread{1.5}
\setlength{\parindent}{16mm}

\titleformat{\section}
  { \filcenter % center
    \fontsize{14pt}{14pt}
    \bfseries % semi bold
    \MakeUppercase
  }
  {\thesection~}
  {0pt}
  {}
\titlespacing*{\section}{0pt}{12pt}{6pt}

\titleformat{\subsection}
  { \fontsize{14pt}{14pt}
  }
  % TODO: sentence case
  {\thesubsection~}
  {0pt}
  {}
\titlespacing*{\subsection}{0pt}{6pt}{0pt}

% all sections will start at new page
\let\oldsection\section
\renewcommand{\section}{\clearpage\oldsection}

\newcommand{\unnumberedSection}[1]{%
  \section*{#1}%
  \addcontentsline{toc}{section}{#1}%
}

\begin{document}


титульний аркуш
\BgThispage
\thispagestyle{empty}

\newpage
завдання
\BgThispage
\section*{реферат}
\BgThispage

% зміст
\tableofcontents

\unnumberedSection{перелік умовних познак, символів, скорочень і термінів}
УДК --- Універсальна десяткова класифікація

НР - наукова робота

% ommited because it's only used for groupping
% основну частину:
  \unnumberedSection{вступ}  
  Задача, яку розглядає дана робота ---
  переклад та перевірка точності перекладу текстів (НР)
  з натуральної мови на формальну (шифр УДК).
  
  Точність УДК шифрів є ключовою для ефективної класифікації
  та пошуку наукової літератури.
  Однак, ручний вибір та перевірка УДК шифрів вимагає значних зусиль,
  займає багато часу та супроводжується високим ризиком виникнення помилок.
  Зовнішня перевірка може зменшити ризик помилок, але вимагає додаткових зусиль.
  Натомість, автоматизоване рішення може значно зменшити час та ресурси,
  необхідні для вибору та перевірки, а також мінімізувати ризик помилок.
  Тому ця робота має на меті розробити надійний
  та ефективний автоматизований інструмент для вибору
  та перевірки УДК шифрів НР.

  Перевірка може бути виконана у два кроки:
  \begin{enumerate}[labelindent=\parindent, leftmargin=*]
    \item Переклад
    \item Порівняння наданого шифру із перекладом.
  \end{enumerate}
  Переклад у свою чергу вимагає попереднього стискання
  вихідного тексту для виключення надмірностей.
  Через розміри та семантику шифрів,
  стисненний результат має представляти собою дуже короткий (декілька речень)
  текст, який перераховує теми оригінального тексту та їх співвідношення.

  % ommited because it's only used for groupping
  % основний текст кваліфікаційного проєкту (роботи):

  \section{збір та аналіз вимог}
  \subsection{висновки}
  \lipsum[1-3]
  \BgThispage

  \section{проєктування}
  \subsection{висновки}
  \BgThispage

  \section{зовнішнє проєктування}
  \subsection{висновки}
  \BgThispage

  \section{внутрішнє проєктування}
  \subsection{висновки}
  \BgThispage

  \section{проєктування архітектури системи}
  \subsection{висновки}
  \BgThispage

  \section{розробка програми}
  \subsection{висновки}
  \BgThispage

  \section{тестування та налагодження}
  \subsection{висновки}
  \BgThispage

  \unnumberedSection{загальні висновки та рекомендації}
  \BgThispage

  \unnumberedSection{список використаної літератури}
  \BgThispage

  \unnumberedSection{додатки}
  \BgThispage

\end{document}
