\documentclass{article}
\usepackage{polyglossia}
\usepackage{titlesec}

\setdefaultlanguage{ukrainian}
\setotherlanguage{english}

\newfontfamily{\cyrillicfont}{Times New Roman}

\let\oldsection\section
\renewcommand{\section}{\clearpage\oldsection}

\setlength{\parskip}{10pt}
\setlength{\parindent}{0pt}

\begin{document}

\section{Title}
Я \textemdash~ Сафонов Данило,\\
тема \textemdash~
розробка засобів підбору та перевірки коректності УДК-шифрів наукових робіт.

\section{Універсальна Десяткова Класифікація 1}
На слайді наведено докази популярності системи УДК.
Ця популярність обгрунтована декількома причинами.

По-перше, ця система намагаєтся охопити усі знання людства, тобто є універсальною
\textemdash~ її можна використовувати у бібліотека з будь-якими типами текстів.

По-друге, завдяки своєму синтаксису вона дозволяє описувати складні відносини
між різними темами, які можуть бути скомбіновані в одному тексті.
Разом із першим пунктом, це дозволяє класифікувати усі можливі тексти.

В третіх, система підтримує більше 50 мов текстів.
Таким чином можна класифікувати будь-які, навіть складні теми,
на великій кількості популярних мов.

Четвертою причиною можна вважати активну підтримку систему.

Загалом, ця система дозволяє організовувати великі об'єми текстів,
і робити це єффективно та стандартизовано, з можливістю подальшого розширення.

\section{Універсальна Десяткова Класифікація 2}
Як я вже казав на попередньому слайді, ця система намагаєтся класифікувати
усі різні типи текстів, та комбінції їх тем.
З цього отримуємо дуже складну систему:
\begin{itemize}
  \item 9 класів, поділені на більш ніж 60,000 підкласів;
  \item 10 синтаксичних поширень та 6 поєднувальних символів,
	  вони дозволяють комбінувати підкласи для текстів які охоплюють декілька тем одразу,
		а це більшість.
\end{itemize}

Крім того, через те що дуже багато текстів насправді відносятся до декількох тем,
шифри текстів можуть виходити дуже довгими.

З цього виходить що система може бути використана тільки тренованими працівниками,
і навіть тоді це будє займати вагомий час.

\subsection{Summary}
Тож ми маємо дуже популярний і сильний інструмент, алє в той самий час,
використання цього інструменту вимагає додаткового тренування та займає багато часу.
Тобто автоматизуючий ці процеси, навіть частково, буде зєкономлено багато ресурсів.

\section{Use-case}
TODO:

\section{Архітектура 1}
TODO:

\section{Архітектура 2}
TODO:

\section{Архітектура 3}
TODO:

\section{Архітектура 4}
TODO:

\section{Архітектура 5}
TODO:

\section{Архітектура 6}
TODO:

\section{Засоби програмування}
TODO:

\section{Вибір формату серіалізації}
TODO:

\section{Алгоритм витягування ключових слів з тексту}
TODO:

\section{Алгоритм перетворення ключових слів на класи УДК}
TODO:

\section{Порівняння списків класів УДК}
TODO:

\end{document}
